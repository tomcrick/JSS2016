%% 
%% Copyright 2007, 2008, 2009 Elsevier Ltd
%% 
%% This file is part of the 'Elsarticle Bundle'.
%% ---------------------------------------------
%% 
%% It may be distributed under the conditions of the LaTeX Project Public
%% License, either version 1.2 of this license or (at your option) any
%% later version.  The latest version of this license is in
%%    http://www.latex-project.org/lppl.txt
%% and version 1.2 or later is part of all distributions of LaTeX
%% version 1999/12/01 or later.
%% 
%% The list of all files belonging to the 'Elsarticle Bundle' is
%% given in the file `manifest.txt'.
%% 

%% Template article for Elsevier's document class `elsarticle'
%% with numbered style bibliographic references
%% SP 2008/03/01

\documentclass[preprint,12pt,authoryear]{elsarticle}

%% Use the option review to obtain double line spacing
%% \documentclass[authoryear,preprint,review,12pt]{elsarticle}

%% Use the options 1p,twocolumn; 3p; 3p,twocolumn; 5p; or 5p,twocolumn
%% for a journal layout:
%% \documentclass[final,1p,times]{elsarticle}
%% \documentclass[final,1p,times,twocolumn]{elsarticle}
%% \documentclass[final,3p,times]{elsarticle}
%% \documentclass[final,3p,times,twocolumn]{elsarticle}
%% \documentclass[final,5p,times]{elsarticle}
%% \documentclass[final,5p,times,twocolumn]{elsarticle}

%% For including figures, graphicx.sty has been loaded in
%% elsarticle.cls. If you prefer to use the old commands
%% please give \usepackage{epsfig}

%% The amssymb package provides various useful mathematical symbols
\usepackage{amssymb}
%% The amsthm package provides extended theorem environments
%% \usepackage{amsthm}
%% The lineno packages adds line numbers. Start line numbering with
%% \begin{linenumbers}, end it with \end{linenumbers}. Or switch it on
%% for the whole article with \linenumbers.
%% \usepackage{lineno}
\usepackage{url}
\usepackage{paralist}
\usepackage{color,soul}

\usepackage{natbib}
% hyperlinks for references
\usepackage[pdftex,plainpages=false]{hyperref}

%% natbib.sty is loaded by default. However, natbib options can be
%% provided with \biboptions{...} command. Following options are
%% valid:
%%   round  -  round parentheses are used (default)
%%   square -  square brackets are used   [option]
%%   curly  -  curly braces are used      {option}
%%   angle  -  angle brackets are used    <option>
%%   semicolon  -  multiple citations separated by semi-colon (default)
%%   colon  - same as semicolon, an earlier confusion
%%   comma  -  separated by comma
%%   authoryear - selects author-year citations (default)
%%   numbers-  selects numerical citations
%%   super  -  numerical citations as superscripts
%%   sort   -  sorts multiple citations according to order in ref. list
%%   sort&compress   -  like sort, but also compresses numerical citations
%%   compress - compresses without sorting
%%   longnamesfirst  -  makes first citation full author list
%%
\biboptions{semicolon,sort&compress}


\journal{Journal of Systems and Software}

\begin{document}

\begin{frontmatter}

%% Title, authors and addresses

%% use the tnoteref command within \title for footnotes;
%% use the tnotetext command for theassociated footnote;
%% use the fnref command within \author or \address for footnotes;
%% use the fntext command for theassociated footnote;
%% use the corref command within \author for corresponding author footnotes;
%% use the cortext command for theassociated footnote;
%% use the ead command for the email address,
%% and the form \ead[url] for the home page:
%% \title{Title\tnoteref{label1}}
%% \tnotetext[label1]{}
%% \author{Name\corref{cor1}\fnref{label2}}
%% \ead{email address}
%% \ead[url]{home page}
%% \fntext[label2]{}
%% \cortext[cor1]{}
%% \address{Address\fnref{label3}}
%% \fntext[label3]{}

% JORS title: ``The Nebuchadnezzar Effect: Dreaming of Sustainable Software through Sustainable Software Architectures''
\title{Sustainable Software $\iff$ Sustainable Software Architectures}

%% use optional labels to link authors explicitly to addresses:
%% \author[label1,label2]{}
%% \address[label1]{}
%% \address[label2]{}

\author[hudd]{Colin C. Venters}
\author[cmu]{Tom Crick\fnref{footnote1}}
\author[csu]{Birgit Penzenstadler}
\address[hudd]{School of Computing \& Engineering, University of
  Huddersfield, UK}
\address[cmu]{Department of Computing \& Information Systems, Cardiff
  Metropolitan University, UK}
\address[csu]{Department of Computer Engineering \& Computer Science, California State University, Long Beach, USA}

% corresponding author -- this can change!
\fntext[footnote1]{Corresponding author at: Department of Computing \&
  Information Systems, Cardiff Metropolitan University, Cardiff CF5 2YB, UK. Tel: +44 (0)2920417174.
\newline
E-mail addresses: \url{c.venters@hud.ac.uk} (C. C. Venters), 
\url{tcrick@cardiffmet.ac.uk} (T. Crick) and 
\url{birgit.penzenstadler@csulb.edu} (B. Penzenstadler)}

% JORS abstract
\begin{abstract}
Sustainability has emerged as a broad issue for society as a result of
significant concerns about the environmental, economic and social
consequences of the rapid consumption of the planets natural
resources, and exponential economic and population
growth. Sustainability is emerging as an area of growing interest in
the field of software engineering as a result of the extent to which
software systems impact many aspects of society both at the macro and
micro-level. While there is no agreed definition of the concept it is
increasingly being considered as a non-functional requirement; a
desired quality of a software system. One of the principal challenges
in defining sustainability as a non-functional requirement is how to
develop appropriate metrics and measures to demonstrate that the
software is sustainable. Software architectures are the foundation of
any software system and provide a mechanism for reasoning about
quality attributes. Given the high dependency of non-functional
requirements on their software architecture, this paper proposes that
sustainable software architectures are fundamental to the development
of technically sustainable software.
\end{abstract}

\begin{keyword}
%% keywords here, in the form: keyword \sep keyword
Software sustainability \sep Software architectures \sep Architectural sustainability
%% PACS codes here, in the form: \PACS code \sep code

% JORS paper keywords:
% Architectural sustainability, non-functional requirements, software
% architectures, software quality, sustainability, software
% sustainability, technical sustainability.


%% MSC codes here, in the form: \MSC code \sep code
%% or \MSC[2008] code \sep code (2000 is the default)

\end{keyword}

\end{frontmatter}

%% \linenumbers

\section*{General Comments}


\begin{itemize}
\item Narrow down pitch for the paper -- tight and concise;
\item Obviously align to journal special issue (lots of themes), but
we are going big picture to align to the call, not narrow and niche to
align to one of the sub-themes;
\item Metrics for architectural sustainability -- how can we start to
identify/frame/measure/capture technical sustainability (and what this
potentially means from the software engineering/architecture
perspective?) e.g. as mentioned as the start of Section 4:
{\emph{Which are the most appropriate architectural-level metrics to
analyze the sustainability of software architectures?}};
\item \hl{All:} Section 4 has the meat of the previous JORS paper -- see what is
there, what is missing, what is needed, etc;
\item \hl{Colin:} will start working on Section 4 and get the main thrust;
\item \hl{All:} try and capture the big messages of the paper;
\item \hl{Colin:} tweak abstract to set the main theme for us to work on;
\end{itemize}

%% main text
\section{Introduction}\label{sec:intro}

Sustainability has been identified as an important future topic in the
field of software engineering as software systems become increasingly
more complex and operate in evolving, distributed
eco-systems~\citep{geist+lucas:2009}. However, the concept of software
sustainability is an elusive and ambiguous term with diametrical
opposed views and interpretations~\citep{venters-et-al:2014}. As a
result, there is a considerable amount of mystification and divergence
regarding what software sustainability means, how it can be measured
or demonstrated, and how to train and educate the broad spectrum of
domain scientists or advance the skills of software engineers to
develop software that is
sustainable~\citep{penzenstadler+fleischmann:2011}.  Change is
inevitable~\citep{bener-et-al:2014}; it is estimated that
approximately 50--70\% of the total lifecycle cost is spent on
evolving a system~\citep{ecklund-et-al:1996}. If change is an
inevitable feature of the software lifecycle this raises the question,
what is the most efficient and effective method or approach for
managing change and evolution in terms of software’s sustainability?

The biblical tale of Nebuchadnezzar's
dream\footnote{\url{https://en.wikipedia.org/wiki/Daniel_2}} relates
to a statue composed of different types of metal built on a foundation
of clay and iron. However, iron and clay are materials that cannot
bond to form a long-lasting foundation and will deteriorate
overtime. The analogy of the feet of clay is now commonly used to
refer to a weakness or flaw. The central thesis of this paper is that
sustainable software should be built from a strong and solid
foundation that allows efficient and effective maintenance and
evolutionary change. To achieve this, this paper proposes that
sustainable software architectures are fundamental to the development
of technically sustainable software. The principle aim of this paper
is to explore existing work to provide the theoretical foundation to
support our thesis. Section~\ref{sec:sussoft} examines the concept of
sustainability and its relationship to sustainable
software. Section~\ref{sec:sust+softarch} examines the relationship
between non-functional requirements, sustainability and software
architectures. Section~\ref{sec:sustsoftarch} explores emerging work
in the area of developing sustainable software architectures. In
Section~\ref{sec:summconc}, conclusions are drawn and future
directions are outlined.


\section{Sustainable Software}\label{sec:sussoft}

Before software sustainability can be measured as an attribute it must
be understood~\citep{seacord-et-al:2003}. The word sustainability is
derived from the Latin {\emph{sustinere}}. The Oxford English
Dictionary defines sustainability as ‘the quality of being sustained’,
where sustained can be defined as ‘capable of being endured’ and
‘capable of being ‘maintained’. This suggests that longevity and the
ability to maintain are key factors at the heart of understanding
sustainability.  As a part of the concept of sustainable development,
the most widely adopted definition of sustainability was that proposed
by the Brundtland Report~\citep{brundtlandreport:1987}, defined as
`{\emph{meeting the needs of the present without compromising the
ability of future generations to meet their own needs}}'. The word
`{\emph{need}}' is central to this definition and includes a dimension
of time, present and future. However, this definition is problematic
for several reasons including that it is broad in its scope, it is
open to interpretation, and it is difficult, if not impossible, to
quantify.

In recent years, a triple bottom line perspective of sustainability
has been adopted which considers sustainability to include three
dimensions: environment, society and economy [10]. Environmental
sustainability is concerned with minimizing the impact on the
environment and natural resources. Social sustainability is concerned
with building social equity. Economic sustainability is concerned with
economic growth and its impact on social or natural resources. It is
argued that by addressing the three dimensions it can lead to more
sustainable outcomes [11]. Goodland [12] and Penzenstadler and Femmer
[13] extend these dimensions to include individual and technical
sustainability where {\emph{individual sustainability}} is `{\emph{the
maintenance of the private good of individual human capital}}'
i.e. health, education etc., and {\emph{technical sustainability}} is
`{\emph{the long-term usage of systems and their adequate evolution
with changing surrounding conditions and respective
requirements}}'. In addition, we can also consider the five dimensions
in relation to three orders of impact or effects of software systems
[10]:

\begin{itemize}
\item {\emph{First-order:}} direct impacts created by the physical
existence and the processes involved including design, production,
distribution, maintenance and disposal;
\item {\emph{Second-order:}} indirect impacts created by ongoing
usage;
\item {\emph{Third-order:}} systemic impacts aggregated over the
medium to long term.
\end{itemize}

Koziolek [14] takes a literal interpretation of the concept of
sustainability and suggests that sustainable software can be defined
as `{\emph{a software-intensive system that operates for more than
fifteen years}}'. This is a position supported by Tamai and Torimitsu
[15] who suggest that the average software lifetime is ten years, with
a minimum of two years, and a maximum of thirty. In addition, they
highlight that longevity requirements can be embedded in a number of
domains where there are large [financial] investments. Koziolek [14]
extends this definition where sustainable software is defined as
`{\emph{a long-living software system which can be cost-efficiently
maintained and evolved over its entire life-cycle}}'. This definition
suggests that maintainability and extensibility are key features of
sustainability, which are tightly coupled with an economical
dimension.

Similarly, a number of definitions have emerged from the field of
software engineering, which focuses on the sustainability of the
software artifact where maintainability and evolution are key factors
of sustainability [7, 16-18]. In this context maintainability can be
defined as `{\emph{the ease with which a software system or component
can be modified to correct faults, improve performance or other
attributes, or adapt to a changed environment}}' [19]. Maintainability
is also related to extendability and flexibility, which the former is
concerned with increasing storage or functional capacity, and the
later is concerned with integration with other applications or
environments respectively. The main objective of software evolution is
ensuring reliability and flexibility of the system where the former is
concerned with functional performance and is also related to
availability. This strongly suggests that software sustainability is a
composite attribute.

There has also been a parallel interest in defining sustainable
software from a software development perspective, which is concerned
with the broader direct and indirect impacts on the economy, society
and the environment [20-21]. However, Fenner et. al., [22] argue that
for sustainable engineering to be successful it requires a paradigm
shift in thinking to embrace a holistic approach enshrined in the
field of complex systems science.

A diametrically opposed position of an absolute definition of software
sustainability is that it is simply an emergent property of a software
system determined by market forces [23]. This suggests that
sustainability cannot be designed or engineered and quantified until
after the software system is operational. While it could be argued
that software sustainability is an emergent property similar to safety
[24], it is a highly probable that a software artifact that has
endured over time will have at the very minimum been maintained
[25]. This suggests that software sustainability is at the very
minimum a measure of maintainability metrics such as the
maintainability index, which can be derived from such measures as
lines-of-code, McCabe or Halstead complexity [26]. In addition, the
software artifact may have evolved or been ported on to different
platforms. This strongly suggests that software sustainability is not
an emergent property and is a composite attribute with a number of
sub-characteristics related to maintenance and evolution.

Increasingly, software sustainability is being considered a first
class, non-functional requirement [24]. In the field of software
engineering, non-functional requirements or software quality
attributes can be defined as `{\emph{the degree to which a system,
component or process meets a stakeholders needs or expectations}}'
[19]. This aligns with the Brundtland [9] definition of sustainability
addressing needs.

Without explicit reference to specific non-functional requirements,
the GREENSOFT model proposed by Naumann et. al., [21] is designed to
incorporate a range of non-functional requirements within the three
categories of the sustainability criteria and metrics section of the
reference model. This separation allows the examination of first-,
second- and third- order impacts from an environmental perspective
that result from effects of supply, effects of usage and systemic
effects. However, they suggest that the fundamental question at the
heart of the model is not, in which phase are metrics applied or in
which phases are they taken in order to improve the quality
attributes? The principal question is thus: {\emph{in which life cycle
phase can the related effects be observed?}}

Venters et. al., [27-28] defined software sustainability as a
composite, non-functional requirement which is `{\emph{a measure of a
systems extensibility, interoperability, maintainability, portability,
reusability, scalability, and usability}}'. Several of the metrics are
directly related to the concept of evolution of the software
system. The rationale for including usability as a metric of
sustainability is that it is directly related to perceived usefulness
from a stakeholder’s perspective and thereby aligns sustainability
with the issue of need. In addition, several of the quality attributes
specify the `{\emph{effort required}}' to achieve a particular
outcome. This suggests that the concept of sustainability is strongly
coupled to other quality attributes such as energy and cost
efficiency, and resource utilization over the software’s entire
lifetime and aligns with the dimensions of environmental and economic
sustainability.

Defining software sustainability as a composite, non-functional
requirement is also a position supported by Calero, Bertoa, and Moraga
[29] who suggest that software sustainability is related to a number
of the main quality attributes and their sub-characteristics defined
in ISO/IEC 25010 [30]; the standard has eight product quality
characteristics and thirty-one sub-characteristics. However, they
suggest that sustainability can be considered from two perspectives:
{\emph{energy efficiency}} and {\emph{perdurability}}. Energy
efficiency is related to consumption and resource optimization, which
aligns to the dimension of environmental sustainability. Based on the
sub-characteristics of {\emph{reusability}}, {\emph{modifiability}},
and {\emph{adaptability}} they define {\emph{perdurability}} as the
`{\emph{degree to which a software product can be modified, adapted
and reused in order to perform specified functions under specified
conditions for a long period of time}}'. However, this significantly
narrows the view of software sustainability as a composite,
non-functional requirement and potentially eliminates important
software quality attributes related to software evolution. Similarly,
it is not clear why attempting to redefine software sustainability in
terms of its perdurability i.e. very durable, is different from the
overall aim of making software sustainable, at least in terms of the
artifact, as the basic definition of sustainability is underpinned by
the idea of enduring. Similarly, Koziolek et. al., [31] define
software sustainability from a maintainability perspective in terms of
its modifiability, reusability, modularity and testability. These
definitions of software sustainability strongly suggest that it can be
categorized as a composite, non-functional requirement. At the very
minimum, technical software sustainability should address two core
quality attributes including appropriate sub-characteristics:
maintainability and extendibility. Nevertheless, maintainability as
defined by ISO/IEC 25010 has new sub-characteristics of modularity,
reusability, and modifiability, which address the issue of software
evolution. As a result, what metrics and measures are suitable to
demonstrate software sustainability is an open research problem.

However, one of the principal challenges in defining software
sustainability as a non-functional requirement is how to demonstrate
that the quality factors have been addressed in a quantifiable
way. How this might be achieved is discussed in the following section.


\section{Sustainability \& Software Architectures}\label{sec:sust+softarch}

To achieve technical sustainable software, we postulate that
[sustainable] software architectures are fundamental to their
development [28]. A software architecture is `{\emph{the fundamental
organization of a system embodied in its components, their
relationships to each other, and to the environment, and the
principles guiding its design and evolution}}' [ISO/IEC
42010-2007]. The rationale for this position is that it is argued that
successful software systems development and evolution is highly
dependent on making informed decisions at the architectural level as
the architecture is the primary carrier of system qualities such as
maintainability, modifiability, reusability, portability and
scalability, none of which it is argued can be achieved without a
unifying architectural vision [33-34]. In addition, Koziolek [35]
argues that software architectures determine sustainability as they
influence how developers are able to understand, analyze, extend, test
and maintain a software system. As a result, software architectures
are not only the blueprint of how the software system will be built,
they hold the key to post-deployment system understanding,
maintenance, and evolution. As a result, this strongly suggests that
software architectures are fundamental to achieving software
sustainability.

An example of how software architectures are fundamental to the
development of sustainable software was reported by Berriman et. al.,
[36]. They present a case study of an approach to sustainable software
applied over a ten-year period to astronomy software services at the
NASA Infrared Processing and Analysis Center at Caltech. In this context,
software sustainability is implicitly defined as long living. Their
approach involved using a component-based architecture, which
consisted of approximately one hundred core components, and was
designed from its inception to support sustainability i.e. longevity,
extensibility, and portability. The rationale for implementing the
architectural style was to enable reusability, adaptability and
portability. Their approach demonstrates that longevity can be
embedded in a software architecture at the outset to support changing
stakeholder requirements and technical obsolesce. In addition, they
recommend a number of best practices for software sustainability based
on their experience:

\begin{itemize}
\item Design for sustainability, extensibility, reusability and
portability from the outset;
\item Adopt a component-based architecture;
\item Assess impact and risk of adopting new or emerging technology;
\item Build a user community;
\item Encourage stakeholder engagement;
\item Adopt rigorous software engineering practices;
\item Develop open source software.
\end{itemize}

However, architectural design is in part a creative process where the
design of a system must provide a balance between the functional and
non-functional requirements. As a result, Sommerville [37] suggests
that it is useful to think of this process as a series of decisions
where a number of fundamental questions are answered rather than a
sequence of activities. In addition, Avgeriou, Stal, and Hilliard [38]
highlight that while the software architecture is the foundation of a
software system encompassing a system architects and stakeholders
strategic decisions, these are often made in an unsystematic and
undocumented manner. As a result, this can lead to architectural drift
and erosion, resulting in a decrease in software quality, which in
turn leads to increased costs and dissatisfied stakeholders. This
suggests the architectural design and reasoning is based on the level
expertise of the software architect and tacit architectural
knowledge. As a result, it is essential to consider the sustainability
of the software architecture as an intrinsic part of overall software
sustainability.


\section{Sustainable Software Architectures}\label{sec:sustsoftarch}

Architecture sustainability is the capacity of a software architecture
to endure different types of change through efficient maintenance and
orderly evolution over its entire life cycle [38]. Koziolek [14]
conducted a systematic review of the literature in order to address a
number of key questions related to architectural sustainability:

\begin{itemize}
\item How do scenario-based architecture evaluation methods used
industry support the sustainability of software architectures?
\item Which are the most appropriate architectural-level metrics to
analyze the sustainability of software architectures?
\end{itemize}

Twenty scenario-based methods were initially identified. Of these only
two, ATAM [39] and ALMA [40], met the criteria for inclusion; were
active and applied in industrial settings. While ATMA was not
specifically designed for sustainability evaluation it offered a range
of techniques in the context of sustainability evaluation. In
contrast, ALMA was specifically designed for modifiability and offers
a number of techniques for change scenario elicitation. However, the
overall results suggest that existing scenario-based methods do not
provide sufficient support for the systematic analysis of ripple
effects, or the integration with reverse engineering tools and
knowledge management support. Similarly, forty architectural-level
metrics were identified, which could potentially assist sustainability
evaluation of implemented architectures. However, many were based on
plausibility and have not been systematically validated. As a result,
their value in addressing sustainability is an open research question.

Zdun et. al., [41] suggest that software architectures not only
comprise a systems structure but essential design decisions based on
architectural knowledge. They argue that to achieve sustainable
architectures, requires capturing significant sustainable design
decisions and their rationale as failure to do so can lead to decision
rationale erosion. They derive five key criteria to define decision
sustainability:

\begin{itemize}
\item Strategic consequences;
\item Measurable and manageable;
\item Achievable and realistic;
\item Rooted in requirements;
\item Timeliness.
\end{itemize}

They suggest that the five criteria are strongly related to the
decision life cycle. As a result, the evolution of decisions across
the life cycle affects the degree of sustainability achieved at any
given time period. Based on the results of a case study, they propose
a set of guidelines to identify and capture a minimalistic set of
relevant decisions and trace links, including:

\begin{itemize}
\item Establishing explicit traceability links between decisions and
requirements;
\item Establishing traceability links among decisions, architecture,
and code;
\item The use and application of design rationale;
\item Minimal decision documentation.
\end{itemize}

As a result, it is suggested that capturing and maintaining
relationships between sources and architectural design decisions can
prevent `architectural knowledge vaporization'. How this can be
achieved in practice is unclear and provides further avenues for
research.

Avgeriou, Stal, and Hilliard [38] identify several causes of change
that are significant for architecture sustainability:

\begin{itemize}
\item New requirements emerge while older requirements change;
\item Interdependence between requirements and architecture;
\item Changes in business strategies and goals;
\item Environment changes;
\item Architecture erosion or drift;
\item Accidental complexity;
\item Technology change;
\item Deferred decisions to meet near-term goals;
\item Human error.
\end{itemize}

To improve architecture sustainability they propose `{\emph{systematic
architecting}}' which considers a system in its total environment, where
environment is defined as the `{\emph{context determining the setting and
circumstances of all influences upon a system including developmental,
technological, business, operational, organizational, political,
economic, legal, regulatory, ecological and social influences}}'
[42]. As a whole, these elements form the basis for establishing the
forces that architects must consider when making decisions and
identifying the risks to be mitigated throughout the systems
lifecycle. To support this, they distinguish between three types of
approaches for handling change systematically which are listed in
order of severity:

\begin{itemize}
\item Refactoring: modifying existing components without impacting
functionality.
\item Renovating: rebuilding one or more essential components from
scratch;
\item Rearchitecting: creating a new architecture.
\end{itemize}

While maintainability and evolution are usually treated individually
they advocate making explicit the differences between the approaches
as they suggest that mixing the three types obfuscates the challenges
of architecture sustainability for practicing architects and offers no
clear direction to researchers.

To address the cost-effective and sustainable evolution of industrial
software systems, Koziolek et. al., [43] proposed MORPHOSIS: a
holistic, sustainable, software architectural method that incorporates
evolution scenario analysis to deal with technological change and
unexpected redesign; architecture enforcement to avoid architectural
erosion; and architectural-level code metrics framework to assess
trends of sustainability. Architectural-level code metrics were based
on existing metrics and selected on the basis of their relevance to
maintainability [44]. They argue that alternative approaches have only
gained limited adoption in practice because the return on investment
is unknown. However, they acknowledge that the limitations of their
approach are that the metrics are not widely used in practice, some
metrics conflict with each other, and that they could not quantify the
return on investment of applying their method.

In a follow on study, Koziolek et. al., [45] argue that it is
difficult to express a software architecture’s sustainability in a
single metric as relevant information may be spread across a range of
related factors including requirements and architecture design
documents, technology choices, source code, systems context, and the
software architectures tacit knowledge. As a result, software
architecture sustainability should consider multiple perspective
including volatile requirements, technology decisions, architecture
erosion, and modularization. Applying the MORPHOSIS approach [46],
they report the results of a two-year longitudinal study, which
tracked selected sustainability measurements of a large-scale,
distributed industrial control system. The results suggest that a
number of predicted evolution scenarios had occurred and those that
had not were still valid. Similarly, the use of architecture
enforcement to avoid architectural erosion created a higher awareness
for the architecture specification resulting in fewer dependency
violations. Finally, the architectural-level code metrics framework
led to an improvement in the overall code quality at the design level
because a measurement instrument was in place. This suggests that
assessments can be conducted with limited effort and that through
regular assessment code can be improved through refactoring to achieve
improved sustainability. Nevertheless, further work is required to
correlate software maintenance costs with the architectural metrics to
enable quantitative cost-benefit analysis. However, it was noted that
software maintenance remained a challenge.

Sehestedt, Cheng, and Bouwers [47] state that software architectures
and their representations in models are instrumental in achieving
sustainability, and the fulfillment of functional and non-functional
requirements. They suggest that the quality of a software
architectural model can be measured by evaluating it against the
following criteria: {\emph{completeness}}, {\emph{consistency}},
{\emph{correctness}} and {\emph{clarity}}. In addition, they propose
seven, system independent metrics, against which the four-C's criteria
can be judged:

\begin{itemize}
\item Decomposition quality;
\item Best practices adherence;
\item View consistency;
\item Rationalization completeness;
\item Requirement fulfillment;
\item Change scenario robustness;
\item Decision traceability.
\end{itemize}

The proposed metrics address quality attributes from three views:
architecture models; architectural decisions; and requirement
specifications. This approach differs from traditional methods as it
indirectly assesses the quality of the architecture through its
documentation. The rationale for this is that architecture models and
related documentation are generally not formal models through which an
architect can evaluate the architectural model in a consistent and
repeatable way. However, they acknowledge that coverage of the
proposed metrics is limited and requires validation to test the limits
of their approach.

In contrast, Ameller et. al., [48] present an interesting counter
argument to our position and the general consensus on the symbiotic
relationship between non-functional requirements and software
architectures. The aim of the study was to investigate how architects
deal with non-functional requirements and focused on four questions:

\begin{itemize}
\item What types of non-functional requirements are relevant to software architects?
\item How are non-functional requirements elicited?
\item How are non-functional requirements documented?
\item How are non-functional requirements validated?
\end{itemize}

Based on semi-structured interviews with thirteen software architects
at twelve software intensive organizations covering a variety of
business and application domains, the results revealed a number of
interesting findings. Firstly, non-functional requirements were
principally defined by architects rather than being driven by
stakeholders needs. Secondly, non-functional requirements are
generally not documented or precise in their representation. Finally,
non-functional requirements were only partially validated, if at
all. Overall, the results suggest that there is a significant mismatch
between software engineering theory and practice. However, Buschmann
et. al., [49] suggests that this apparent mismatch provides valuable
insights into how to deal with non-functional requirements in software
development; a prime indicator is their business value. As a result,
software architects need to be pragmatic in how they balance the
inclusion of non-functional requirements as a prime driver of
architecture design.


\section{Summary \& Conclusions}\label{sec:summconc}

In this paper we propose that sustainable software architectures are
fundamental to the development of technical sustainable software as a
basis for discussion in order to consider how we can address the
challenge of developing and achieving sustainable software. The paper
explores previous research to provide the theoretical foundation to
support our thesis.

Examination of the concept of sustainability and its relationship to
software demonstrates that software sustainability has been defined
from a number of different perspectives including that of the software
artifact and the software engineering development process. While there
is no agreement on an absolute definition of software sustainability
there is growing consensus that software sustainability should be
considered a first-class, non-functional requirement that is a measure
of a number of core quality attributes. We suggest that at the very
minimum, a software's technical sustainability should address two core
quality attributes: maintainability and extendibility. To what extent
existing metrics and measures of quality attributes defined within
existing standards are appropriate for measuring a software artifact’s
technical sustainability is an open research question and provides
further avenues for research. In addition, how to make software
sustainable both in terms of the software artifact, the development
process, and how these relate to the wider concerns of environmental,
economic, social, individual, and technical sustainability remains an
open area of research.

Software architectures can be considered the Quoins of sustainable
software. While research into the relationship between software
architectures and sustainability is strictly limited, emerging
evidence suggests that the architecture plays a critical role in
satisfying non-functional requirements. As a result, software
architectures are not only fundamental in understanding how the
software system will be built in the first instance, they are critical
to post-deployment system maintenance and evolution, which in turn
leads to software that is sustainable. While it is suggested that
architectural design is more of a creative process, which requires a
degree of expertise, the principal challenge is how to embed software
architectural practice into software engineering best practice rather
than viewed as a by-product of the software engineering process; which
is particularly true of software developed in academic environments.

An emerging area of interest in the field of software architectures is
architectural sustainability, which aims to address architectural
drift and erosion that can result in a decrease in software
quality. While a small number of approaches have been proposed their
value is unknown and require validation to test their limits. Critical
to architectural sustainability is capturing decision viewpoints and
their rationale as first-class elements of architectural
descriptions. How this can be achieved in practice is unclear and is
an area ripe for research. However, based on our previous work into
the relationship between trust, provenance, and high-value decision
making in data intensive, service-oriented computing environments, we
suggest that integrating requirements traceability methods and
provenance potentially provides an avenue for further research.

If we accept the premise that software architectures are the key to
developing technical sustainable software then ensuring architectural
sustainability is critical to its success. However, the software
engineering profession currently lacks a common ground that
articulates its role in sustainability. Efforts by the Karlskrona
consortium have attempted to address this by articulating the
fundamental principles underpinning design and engineering choices
that affect sustainability within and beyond the software engineering
community [50, 51]. The concept of sustainability is inherently
multi-disciplinary as it concerns a complex system having multiple
dimensions such as environmental, economic, societal, technological
and other perspectives. Our vision is of a future in which a set of
principles and methods inform sustainable software engineering. As a
result, any effort in addressing sustainability involves integrating
concepts, principles, and methods from a range of disciplines and
greater engagement with the communities investigating sustainability.

For sustainability to become a first-class quality consideration, we
intend to progress the research agenda by investigating the
development of a non-functional requirements framework to create a
hierarchically structured set of characteristics including their
metrics and measures for sustainability, and to investigate the role
of software architectures in pre-system and post-system deployment
reasoning about sustainability as a NFR to develop a software
sustainability evaluation framework that will assist in facilitating a
greater holistic view of sustainability.


% \section{Acknowledgements}\label{sec:ack}
% T. Crick acknowledge support from the Software Sustainability
% Institute\footnote{\url{http://www.software.ac.uk}} as a 2014 Fellow.

%% The Appendices part is started with the command \appendix;
%% appendix sections are then done as normal sections
%% \appendix

% JSS asks for a 100 word biog from each of us...
% This is supposed to be typeset after the refs, but can't see how to
% do it in the elsarticle class?
\section*{Author Biographies}

\noindent {\textbf{Colin C. Venters}} is...\\

\noindent {\textbf{Tom Crick}} is Professor of Computer Science \& Public Policy
at Cardiff Metropolitan University, UK. He is the Nesta Data Science
Fellow and a 2014 Fellow of the Software Sustainability
Institute. His research interests...\\

\noindent {\textbf{Birgit Penzenstadler}} is...


\section*{References}\label{sec:refs}

\bibliographystyle{elsarticle-harv} 
\bibliography{jss2016}

%% else use the following coding to input the bibitems directly in the
%% TeX file.

% \begin{thebibliography}{00}

% %% \bibitem{label}
% %% Text of bibliographic item

% \bibitem{}

% \end{thebibliography}
\end{document}
\endinput
%%
%% End of file `elsarticle-template-num.tex'.
